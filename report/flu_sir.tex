\documentclass[12pt]{article}

% Tipografiado y preferencias regionales
\usepackage[utf8]{inputenc}
\usepackage[spanish,es-noindentfirst]{babel}
\usepackage[T1]{fontenc}
\usepackage{lmodern}
\usepackage{csquotes,textcomp,microtype}
\decimalpoint
\unaccentedoperators

% Matemáticas
\usepackage{amsmath}
\usepackage{amsfonts}
\usepackage{amssymb}
\usepackage{amsthm}

% Bibliografía
%\usepackage[style=apa]{biblatex}
%\DeclareLanguageMapping{spanish}{spanish-apa}
%\addbibresource{.bib}

% Formato
\usepackage[hmargin=3cm,vmargin=2.5cm]{geometry}

% Imágenes
\usepackage{tikz}
\usepackage{graphicx}
\graphicspath{../figures/}

% Enlaces
\usepackage[pdftex]{hyperref}
\usepackage[spanish]{cleveref}

% Código
\usepackage{listings}
\usepackage{color}

\lstset{
	language=R,				% the language of the code
	basicstyle=\small\ttfamily,		% the size of the fonts that are used for the code
	numbers=left,					% where to put the line-numbers
	numberstyle=\tiny\color{Blue},	% the style that is used for the line-numbers
	stepnumber=1,					% the step between two line-numbers. If it is 1, each line
	% will be numbered
	numbersep=5pt,                  % how far the line-numbers are from the code
	backgroundcolor=\color{white},  % choose the background color. You must add \usepackage{color}
	showspaces=false,               % show spaces adding particular underscores
	showstringspaces=false,         % underline spaces within strings
	showtabs=false,                 % show tabs within strings adding particular underscores
	frame=single,                   % adds a frame around the code
	rulecolor=\color{black},        % if not set, the frame-color may be changed on line-breaks within not-black text (e.g. commens (green here))
	tabsize=2,                      % sets default tabsize to 2 spaces
	captionpos=b,                   % sets the caption-position to bottom
	breaklines=true,                % sets automatic line breaking
	breakatwhitespace=false,        % sets if automatic breaks should only happen at whitespace
	keywordstyle=\color{RoyalBlue},      % keyword style
	commentstyle=\color{YellowGreen},   % comment style
	stringstyle=\color{ForestGreen}      % string literal style
}

% Comandos
\newcommand{\absdiv}[1]{%
  \par\addvspace{.5\baselineskip}% adjust to suit
  \noindent\textbf{#1}\quad\ignorespaces
}

\newcommand{\R}{\mathbb{R}}

\begin{document}
	\title{Estimación}
	\author{Jorge Esteban Mendoza Ortiz (418002863)\thanks{\href{mailto:esteban.mendoza@ciencias.unam.mx}{\nolinkurl{esteban.mendoza@ciencias.unam.mx}}} \\
	Enrique Moctezuma González (315019344)\thanks{\href{mailto:enriquemg@ciencias.unam.mx}{\nolinkurl{enriquemg@ciencias.unam.mx}}}\\
	Ana Victoria López Rodríguez (314614300)\thanks{\href{anaviclopez@ciencias.unam.mx}{\nolinkurl{anaviclopez@ciencias.unam.mx}}}}
	\date{}
	\maketitle
	
	\begin{abstract}
	\absdiv{sub\_abstract}
	\end{abstract}
	
	% Introducción
	\noindent
	
	\section{Materiales y métodos}
	\subsection*{Fuente de datos}
	
	\subsection*{Descripción del modelo SIR}
	
	\subsection*{Aplicación del modelo}
	
	\section{Resultados}
	
	\begin{figure}
		\centering
		% Created by tikzDevice version 0.12.3 on 2019-08-24 15:52:19
% !TEX encoding = UTF-8 Unicode
\begin{tikzpicture}[x=1pt,y=1pt]
\definecolor{fillColor}{RGB}{255,255,255}
\path[use as bounding box,fill=fillColor,fill opacity=0.00] (0,0) rectangle (433.62,271.01);
\begin{scope}
\path[clip] (  0.00,  0.00) rectangle (433.62,271.01);
\definecolor{drawColor}{RGB}{255,255,255}
\definecolor{fillColor}{RGB}{255,255,255}

\path[draw=drawColor,line width= 0.6pt,line join=round,line cap=round,fill=fillColor] (  0.00,  0.00) rectangle (433.62,271.01);
\end{scope}
\begin{scope}
\path[clip] ( 36.11, 30.69) rectangle (347.26,265.51);
\definecolor{fillColor}{gray}{0.92}

\path[fill=fillColor] ( 36.11, 30.69) rectangle (347.26,265.51);
\definecolor{drawColor}{RGB}{255,255,255}

\path[draw=drawColor,line width= 0.3pt,line join=round] ( 36.11, 67.08) --
	(347.26, 67.08);

\path[draw=drawColor,line width= 0.3pt,line join=round] ( 36.11,118.80) --
	(347.26,118.80);

\path[draw=drawColor,line width= 0.3pt,line join=round] ( 36.11,170.53) --
	(347.26,170.53);

\path[draw=drawColor,line width= 0.3pt,line join=round] ( 36.11,222.25) --
	(347.26,222.25);

\path[draw=drawColor,line width= 0.3pt,line join=round] ( 38.47, 30.69) --
	( 38.47,265.51);

\path[draw=drawColor,line width= 0.3pt,line join=round] (120.97, 30.69) --
	(120.97,265.51);

\path[draw=drawColor,line width= 0.3pt,line join=round] (209.36, 30.69) --
	(209.36,265.51);

\path[draw=drawColor,line width= 0.3pt,line join=round] (297.76, 30.69) --
	(297.76,265.51);

\path[draw=drawColor,line width= 0.6pt,line join=round] ( 36.11, 41.22) --
	(347.26, 41.22);

\path[draw=drawColor,line width= 0.6pt,line join=round] ( 36.11, 92.94) --
	(347.26, 92.94);

\path[draw=drawColor,line width= 0.6pt,line join=round] ( 36.11,144.67) --
	(347.26,144.67);

\path[draw=drawColor,line width= 0.6pt,line join=round] ( 36.11,196.39) --
	(347.26,196.39);

\path[draw=drawColor,line width= 0.6pt,line join=round] ( 36.11,248.12) --
	(347.26,248.12);

\path[draw=drawColor,line width= 0.6pt,line join=round] ( 79.72, 30.69) --
	( 79.72,265.51);

\path[draw=drawColor,line width= 0.6pt,line join=round] (162.22, 30.69) --
	(162.22,265.51);

\path[draw=drawColor,line width= 0.6pt,line join=round] (256.51, 30.69) --
	(256.51,265.51);

\path[draw=drawColor,line width= 0.6pt,line join=round] (339.01, 30.69) --
	(339.01,265.51);
\definecolor{drawColor}{RGB}{248,118,109}
\definecolor{fillColor}{RGB}{248,118,109}

\path[draw=drawColor,line width= 0.4pt,line join=round,line cap=round,fill=fillColor] ( 50.25, 41.73) circle (  1.96);

\path[draw=drawColor,line width= 0.4pt,line join=round,line cap=round,fill=fillColor] ( 56.15, 42.25) circle (  1.96);

\path[draw=drawColor,line width= 0.4pt,line join=round,line cap=round,fill=fillColor] ( 62.04, 43.80) circle (  1.96);

\path[draw=drawColor,line width= 0.4pt,line join=round,line cap=round,fill=fillColor] ( 67.93, 44.84) circle (  1.96);

\path[draw=drawColor,line width= 0.4pt,line join=round,line cap=round,fill=fillColor] ( 73.83, 43.28) circle (  1.96);

\path[draw=drawColor,line width= 0.4pt,line join=round,line cap=round,fill=fillColor] ( 79.72, 46.39) circle (  1.96);

\path[draw=drawColor,line width= 0.4pt,line join=round,line cap=round,fill=fillColor] ( 85.61, 46.39) circle (  1.96);

\path[draw=drawColor,line width= 0.4pt,line join=round,line cap=round,fill=fillColor] ( 91.51, 45.87) circle (  1.96);

\path[draw=drawColor,line width= 0.4pt,line join=round,line cap=round,fill=fillColor] ( 97.40, 42.77) circle (  1.96);

\path[draw=drawColor,line width= 0.4pt,line join=round,line cap=round,fill=fillColor] (103.29, 43.28) circle (  1.96);

\path[draw=drawColor,line width= 0.4pt,line join=round,line cap=round,fill=fillColor] (109.18, 46.90) circle (  1.96);

\path[draw=drawColor,line width= 0.4pt,line join=round,line cap=round,fill=fillColor] (115.08, 43.80) circle (  1.96);

\path[draw=drawColor,line width= 0.4pt,line join=round,line cap=round,fill=fillColor] (120.97, 44.84) circle (  1.96);

\path[draw=drawColor,line width= 0.4pt,line join=round,line cap=round,fill=fillColor] (126.86, 43.28) circle (  1.96);

\path[draw=drawColor,line width= 0.4pt,line join=round,line cap=round,fill=fillColor] (132.76, 43.28) circle (  1.96);

\path[draw=drawColor,line width= 0.4pt,line join=round,line cap=round,fill=fillColor] (138.65, 43.28) circle (  1.96);

\path[draw=drawColor,line width= 0.4pt,line join=round,line cap=round,fill=fillColor] (144.54, 46.90) circle (  1.96);

\path[draw=drawColor,line width= 0.4pt,line join=round,line cap=round,fill=fillColor] (150.43, 50.01) circle (  1.96);

\path[draw=drawColor,line width= 0.4pt,line join=round,line cap=round,fill=fillColor] (156.33, 54.66) circle (  1.96);

\path[draw=drawColor,line width= 0.4pt,line join=round,line cap=round,fill=fillColor] (162.22, 51.56) circle (  1.96);

\path[draw=drawColor,line width= 0.4pt,line join=round,line cap=round,fill=fillColor] (168.11, 47.42) circle (  1.96);

\path[draw=drawColor,line width= 0.4pt,line join=round,line cap=round,fill=fillColor] (174.01, 54.66) circle (  1.96);

\path[draw=drawColor,line width= 0.4pt,line join=round,line cap=round,fill=fillColor] (179.90, 57.77) circle (  1.96);

\path[draw=drawColor,line width= 0.4pt,line join=round,line cap=round,fill=fillColor] (185.79, 63.97) circle (  1.96);

\path[draw=drawColor,line width= 0.4pt,line join=round,line cap=round,fill=fillColor] (191.69, 98.11) circle (  1.96);

\path[draw=drawColor,line width= 0.4pt,line join=round,line cap=round,fill=fillColor] (197.58, 99.15) circle (  1.96);

\path[draw=drawColor,line width= 0.4pt,line join=round,line cap=round,fill=fillColor] (203.47,119.32) circle (  1.96);

\path[draw=drawColor,line width= 0.4pt,line join=round,line cap=round,fill=fillColor] (209.36,157.60) circle (  1.96);

\path[draw=drawColor,line width= 0.4pt,line join=round,line cap=round,fill=fillColor] (215.26,185.01) circle (  1.96);

\path[draw=drawColor,line width= 0.4pt,line join=round,line cap=round,fill=fillColor] (221.15,204.67) circle (  1.96);

\path[draw=drawColor,line width= 0.4pt,line join=round,line cap=round,fill=fillColor] (227.04,245.53) circle (  1.96);

\path[draw=drawColor,line width= 0.4pt,line join=round,line cap=round,fill=fillColor] (232.94,253.29) circle (  1.96);

\path[draw=drawColor,line width= 0.4pt,line join=round,line cap=round,fill=fillColor] (238.83,198.46) circle (  1.96);

\path[draw=drawColor,line width= 0.4pt,line join=round,line cap=round,fill=fillColor] (244.72,185.01) circle (  1.96);

\path[draw=drawColor,line width= 0.4pt,line join=round,line cap=round,fill=fillColor] (250.62,158.63) circle (  1.96);

\path[draw=drawColor,line width= 0.4pt,line join=round,line cap=round,fill=fillColor] (256.51,148.29) circle (  1.96);

\path[draw=drawColor,line width= 0.4pt,line join=round,line cap=round,fill=fillColor] (262.40,136.91) circle (  1.96);

\path[draw=drawColor,line width= 0.4pt,line join=round,line cap=round,fill=fillColor] (268.29,149.84) circle (  1.96);

\path[draw=drawColor,line width= 0.4pt,line join=round,line cap=round,fill=fillColor] (274.19,161.22) circle (  1.96);

\path[draw=drawColor,line width= 0.4pt,line join=round,line cap=round,fill=fillColor] (280.08,159.15) circle (  1.96);

\path[draw=drawColor,line width= 0.4pt,line join=round,line cap=round,fill=fillColor] (285.97,161.22) circle (  1.96);

\path[draw=drawColor,line width= 0.4pt,line join=round,line cap=round,fill=fillColor] (291.87,131.73) circle (  1.96);

\path[draw=drawColor,line width= 0.4pt,line join=round,line cap=round,fill=fillColor] (297.76,126.56) circle (  1.96);

\path[draw=drawColor,line width= 0.4pt,line join=round,line cap=round,fill=fillColor] (303.65,112.60) circle (  1.96);

\path[draw=drawColor,line width= 0.4pt,line join=round,line cap=round,fill=fillColor] (309.54,112.60) circle (  1.96);

\path[draw=drawColor,line width= 0.4pt,line join=round,line cap=round,fill=fillColor] (315.44,124.49) circle (  1.96);

\path[draw=drawColor,line width= 0.4pt,line join=round,line cap=round,fill=fillColor] (321.33,110.01) circle (  1.96);

\path[draw=drawColor,line width= 0.4pt,line join=round,line cap=round,fill=fillColor] (327.22,100.70) circle (  1.96);

\path[draw=drawColor,line width= 0.4pt,line join=round,line cap=round,fill=fillColor] (333.12, 91.39) circle (  1.96);
\definecolor{drawColor}{RGB}{0,191,196}
\definecolor{fillColor}{RGB}{0,191,196}

\path[draw=drawColor,line width= 0.4pt,line join=round,line cap=round,fill=fillColor] ( 56.15, 41.36) circle (  1.96);

\path[draw=drawColor,line width= 0.4pt,line join=round,line cap=round,fill=fillColor] ( 62.04, 41.40) circle (  1.96);

\path[draw=drawColor,line width= 0.4pt,line join=round,line cap=round,fill=fillColor] ( 67.93, 41.45) circle (  1.96);

\path[draw=drawColor,line width= 0.4pt,line join=round,line cap=round,fill=fillColor] ( 73.83, 41.52) circle (  1.96);

\path[draw=drawColor,line width= 0.4pt,line join=round,line cap=round,fill=fillColor] ( 79.72, 41.61) circle (  1.96);

\path[draw=drawColor,line width= 0.4pt,line join=round,line cap=round,fill=fillColor] ( 85.61, 41.72) circle (  1.96);

\path[draw=drawColor,line width= 0.4pt,line join=round,line cap=round,fill=fillColor] ( 91.51, 41.87) circle (  1.96);

\path[draw=drawColor,line width= 0.4pt,line join=round,line cap=round,fill=fillColor] ( 97.40, 42.06) circle (  1.96);

\path[draw=drawColor,line width= 0.4pt,line join=round,line cap=round,fill=fillColor] (103.29, 42.30) circle (  1.96);

\path[draw=drawColor,line width= 0.4pt,line join=round,line cap=round,fill=fillColor] (109.18, 42.61) circle (  1.96);

\path[draw=drawColor,line width= 0.4pt,line join=round,line cap=round,fill=fillColor] (115.08, 43.01) circle (  1.96);

\path[draw=drawColor,line width= 0.4pt,line join=round,line cap=round,fill=fillColor] (120.97, 43.52) circle (  1.96);

\path[draw=drawColor,line width= 0.4pt,line join=round,line cap=round,fill=fillColor] (126.86, 44.18) circle (  1.96);

\path[draw=drawColor,line width= 0.4pt,line join=round,line cap=round,fill=fillColor] (132.76, 45.03) circle (  1.96);

\path[draw=drawColor,line width= 0.4pt,line join=round,line cap=round,fill=fillColor] (138.65, 46.12) circle (  1.96);

\path[draw=drawColor,line width= 0.4pt,line join=round,line cap=round,fill=fillColor] (144.54, 47.52) circle (  1.96);

\path[draw=drawColor,line width= 0.4pt,line join=round,line cap=round,fill=fillColor] (150.43, 49.33) circle (  1.96);

\path[draw=drawColor,line width= 0.4pt,line join=round,line cap=round,fill=fillColor] (156.33, 51.65) circle (  1.96);

\path[draw=drawColor,line width= 0.4pt,line join=round,line cap=round,fill=fillColor] (162.22, 54.64) circle (  1.96);

\path[draw=drawColor,line width= 0.4pt,line join=round,line cap=round,fill=fillColor] (168.11, 58.48) circle (  1.96);

\path[draw=drawColor,line width= 0.4pt,line join=round,line cap=round,fill=fillColor] (174.01, 63.41) circle (  1.96);

\path[draw=drawColor,line width= 0.4pt,line join=round,line cap=round,fill=fillColor] (179.90, 69.77) circle (  1.96);

\path[draw=drawColor,line width= 0.4pt,line join=round,line cap=round,fill=fillColor] (185.79, 77.94) circle (  1.96);

\path[draw=drawColor,line width= 0.4pt,line join=round,line cap=round,fill=fillColor] (191.69, 88.46) circle (  1.96);

\path[draw=drawColor,line width= 0.4pt,line join=round,line cap=round,fill=fillColor] (197.58,101.97) circle (  1.96);

\path[draw=drawColor,line width= 0.4pt,line join=round,line cap=round,fill=fillColor] (203.47,119.35) circle (  1.96);

\path[draw=drawColor,line width= 0.4pt,line join=round,line cap=round,fill=fillColor] (209.36,141.72) circle (  1.96);

\path[draw=drawColor,line width= 0.4pt,line join=round,line cap=round,fill=fillColor] (215.26,170.49) circle (  1.96);

\path[draw=drawColor,line width= 0.4pt,line join=round,line cap=round,fill=fillColor] (221.15,207.49) circle (  1.96);

\path[draw=drawColor,line width= 0.4pt,line join=round,line cap=round,fill=fillColor] (227.04,254.84) circle (  1.96);

\path[draw=drawColor,line width= 0.4pt,line join=round,line cap=round,fill=fillColor] (232.94,243.08) circle (  1.96);

\path[draw=drawColor,line width= 0.4pt,line join=round,line cap=round,fill=fillColor] (238.83,230.95) circle (  1.96);

\path[draw=drawColor,line width= 0.4pt,line join=round,line cap=round,fill=fillColor] (244.72,219.54) circle (  1.96);

\path[draw=drawColor,line width= 0.4pt,line join=round,line cap=round,fill=fillColor] (250.62,208.83) circle (  1.96);

\path[draw=drawColor,line width= 0.4pt,line join=round,line cap=round,fill=fillColor] (256.51,198.75) circle (  1.96);

\path[draw=drawColor,line width= 0.4pt,line join=round,line cap=round,fill=fillColor] (262.40,189.28) circle (  1.96);

\path[draw=drawColor,line width= 0.4pt,line join=round,line cap=round,fill=fillColor] (268.29,180.38) circle (  1.96);

\path[draw=drawColor,line width= 0.4pt,line join=round,line cap=round,fill=fillColor] (274.19,172.02) circle (  1.96);

\path[draw=drawColor,line width= 0.4pt,line join=round,line cap=round,fill=fillColor] (280.08,164.16) circle (  1.96);

\path[draw=drawColor,line width= 0.4pt,line join=round,line cap=round,fill=fillColor] (285.97,156.77) circle (  1.96);

\path[draw=drawColor,line width= 0.4pt,line join=round,line cap=round,fill=fillColor] (291.87,149.82) circle (  1.96);

\path[draw=drawColor,line width= 0.4pt,line join=round,line cap=round,fill=fillColor] (297.76,143.29) circle (  1.96);

\path[draw=drawColor,line width= 0.4pt,line join=round,line cap=round,fill=fillColor] (303.65,137.16) circle (  1.96);

\path[draw=drawColor,line width= 0.4pt,line join=round,line cap=round,fill=fillColor] (309.54,131.39) circle (  1.96);

\path[draw=drawColor,line width= 0.4pt,line join=round,line cap=round,fill=fillColor] (315.44,125.97) circle (  1.96);

\path[draw=drawColor,line width= 0.4pt,line join=round,line cap=round,fill=fillColor] (321.33,120.88) circle (  1.96);

\path[draw=drawColor,line width= 0.4pt,line join=round,line cap=round,fill=fillColor] (327.22,116.09) circle (  1.96);

\path[draw=drawColor,line width= 0.4pt,line join=round,line cap=round,fill=fillColor] (333.12,111.59) circle (  1.96);
\end{scope}
\begin{scope}
\path[clip] (  0.00,  0.00) rectangle (433.62,271.01);
\definecolor{drawColor}{gray}{0.30}

\node[text=drawColor,anchor=base east,inner sep=0pt, outer sep=0pt, scale=  0.88] at ( 31.16, 38.18) {0};

\node[text=drawColor,anchor=base east,inner sep=0pt, outer sep=0pt, scale=  0.88] at ( 31.16, 89.91) {100};

\node[text=drawColor,anchor=base east,inner sep=0pt, outer sep=0pt, scale=  0.88] at ( 31.16,141.64) {200};

\node[text=drawColor,anchor=base east,inner sep=0pt, outer sep=0pt, scale=  0.88] at ( 31.16,193.36) {300};

\node[text=drawColor,anchor=base east,inner sep=0pt, outer sep=0pt, scale=  0.88] at ( 31.16,245.09) {400};
\end{scope}
\begin{scope}
\path[clip] (  0.00,  0.00) rectangle (433.62,271.01);
\definecolor{drawColor}{gray}{0.20}

\path[draw=drawColor,line width= 0.6pt,line join=round] ( 33.36, 41.22) --
	( 36.11, 41.22);

\path[draw=drawColor,line width= 0.6pt,line join=round] ( 33.36, 92.94) --
	( 36.11, 92.94);

\path[draw=drawColor,line width= 0.6pt,line join=round] ( 33.36,144.67) --
	( 36.11,144.67);

\path[draw=drawColor,line width= 0.6pt,line join=round] ( 33.36,196.39) --
	( 36.11,196.39);

\path[draw=drawColor,line width= 0.6pt,line join=round] ( 33.36,248.12) --
	( 36.11,248.12);
\end{scope}
\begin{scope}
\path[clip] (  0.00,  0.00) rectangle (433.62,271.01);
\definecolor{drawColor}{gray}{0.20}

\path[draw=drawColor,line width= 0.6pt,line join=round] ( 79.72, 27.94) --
	( 79.72, 30.69);

\path[draw=drawColor,line width= 0.6pt,line join=round] (162.22, 27.94) --
	(162.22, 30.69);

\path[draw=drawColor,line width= 0.6pt,line join=round] (256.51, 27.94) --
	(256.51, 30.69);

\path[draw=drawColor,line width= 0.6pt,line join=round] (339.01, 27.94) --
	(339.01, 30.69);
\end{scope}
\begin{scope}
\path[clip] (  0.00,  0.00) rectangle (433.62,271.01);
\definecolor{drawColor}{gray}{0.30}

\node[text=drawColor,anchor=base,inner sep=0pt, outer sep=0pt, scale=  0.88] at ( 79.72, 19.68) {abr. 01};

\node[text=drawColor,anchor=base,inner sep=0pt, outer sep=0pt, scale=  0.88] at (162.22, 19.68) {abr. 15};

\node[text=drawColor,anchor=base,inner sep=0pt, outer sep=0pt, scale=  0.88] at (256.51, 19.68) {may. 01};

\node[text=drawColor,anchor=base,inner sep=0pt, outer sep=0pt, scale=  0.88] at (339.01, 19.68) {may. 15};
\end{scope}
\begin{scope}
\path[clip] (  0.00,  0.00) rectangle (433.62,271.01);
\definecolor{drawColor}{RGB}{0,0,0}

\node[text=drawColor,anchor=base,inner sep=0pt, outer sep=0pt, scale=  1.10] at (191.69,  7.64) {Fecha};
\end{scope}
\begin{scope}
\path[clip] (  0.00,  0.00) rectangle (433.62,271.01);
\definecolor{drawColor}{RGB}{0,0,0}

\node[text=drawColor,rotate= 90.00,anchor=base,inner sep=0pt, outer sep=0pt, scale=  1.10] at ( 13.08,148.10) {Casos confirmados};
\end{scope}
\begin{scope}
\path[clip] (  0.00,  0.00) rectangle (433.62,271.01);
\definecolor{fillColor}{RGB}{255,255,255}

\path[fill=fillColor] (358.26,120.54) rectangle (428.12,175.66);
\end{scope}
\begin{scope}
\path[clip] (  0.00,  0.00) rectangle (433.62,271.01);
\definecolor{drawColor}{RGB}{0,0,0}

\node[text=drawColor,anchor=base west,inner sep=0pt, outer sep=0pt, scale=  1.10] at (363.76,161.52) {Datos};
\end{scope}
\begin{scope}
\path[clip] (  0.00,  0.00) rectangle (433.62,271.01);
\definecolor{drawColor}{RGB}{255,255,255}
\definecolor{fillColor}{gray}{0.95}

\path[draw=drawColor,line width= 0.6pt,line join=round,line cap=round,fill=fillColor] (363.76,140.49) rectangle (378.21,154.95);
\end{scope}
\begin{scope}
\path[clip] (  0.00,  0.00) rectangle (433.62,271.01);
\definecolor{drawColor}{RGB}{248,118,109}
\definecolor{fillColor}{RGB}{248,118,109}

\path[draw=drawColor,line width= 0.4pt,line join=round,line cap=round,fill=fillColor] (370.99,147.72) circle (  1.96);
\end{scope}
\begin{scope}
\path[clip] (  0.00,  0.00) rectangle (433.62,271.01);
\definecolor{drawColor}{RGB}{255,255,255}
\definecolor{fillColor}{gray}{0.95}

\path[draw=drawColor,line width= 0.6pt,line join=round,line cap=round,fill=fillColor] (363.76,126.04) rectangle (378.21,140.49);
\end{scope}
\begin{scope}
\path[clip] (  0.00,  0.00) rectangle (433.62,271.01);
\definecolor{drawColor}{RGB}{0,191,196}
\definecolor{fillColor}{RGB}{0,191,196}

\path[draw=drawColor,line width= 0.4pt,line join=round,line cap=round,fill=fillColor] (370.99,133.26) circle (  1.96);
\end{scope}
\begin{scope}
\path[clip] (  0.00,  0.00) rectangle (433.62,271.01);
\definecolor{drawColor}{RGB}{0,0,0}

\node[text=drawColor,anchor=base west,inner sep=0pt, outer sep=0pt, scale=  0.88] at (383.71,144.69) {Originales};
\end{scope}
\begin{scope}
\path[clip] (  0.00,  0.00) rectangle (433.62,271.01);
\definecolor{drawColor}{RGB}{0,0,0}

\node[text=drawColor,anchor=base west,inner sep=0pt, outer sep=0pt, scale=  0.88] at (383.71,130.23) {Simulados};
\end{scope}
\end{tikzpicture}

		\caption{Sample output from tikzDevice}
		\label{plot:test}
	\end{figure}
	
	\section{Discusión}
	
	
	
\end{document}
